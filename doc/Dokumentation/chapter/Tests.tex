\section{Tests}
\label{sec:Tests}
Wir haben das Standard Testframework von Rails verwendet um umfangreiche Unit Tests zu schreiben. Dies basiert auf der Idee des Test-Driven-Development. Also zuerst Tests schrieben, schauen wie sie fehlschlagen und dann die Funktionalit�t implementieren und pr�fen ob der Test erfolgreich ist. ie Unit Tests sichern unsere Grundfunktionalit�t. Des weiteren haben wir mit Cucumber Tests geschrieben. Mit Cucumber (Behavior-Driven-Development) ist es m�glich Tests in Prosaform zu schrieben. Das macht die Tests besser lesbarer auch f�r nicht Informatiker. Die Tests werden h�ufig als User Stories geschrieben in der Form: ''Als <Rolle>, m�chte ich <Ziel/Wunsch>, um <Nutzen>''. Umfangreiche Tests wie sie in einem realen Projekt gemacht werden sollten konnten wir aufgrund der beschr�nkten Ressourcen jedoch nicht durchf�hren. Rails eignet jedoch wunderbar f�r ein Test-Driven-Development da es ein Standard Testframwork integriert. 
Hier ist exemplarisch ein Unit Tests f�r das Model User zu sehen. So wird z.B. auf korrekte Form der eMail Adresse getestet und ob die Attribute Name, Vorname, eMail und Passwort gesetzt sind.
\lstinputlisting[language=Ruby]{chapter/test_user.rb}
Hier ist zu sehen wie Tests mit Cucumber geschrieben werden. Zuerst werden sogenannte Features definiert. Diese sind in der <Gegeben> <Eregnis> <Ergebnis> geschrieben. Die feature sind dabei an User Stories angelegt. Diese Art der Tests sind leicht nachzuvollziehen. So k�nnen z.B. auch die Auftrageber eines Projekts welche evtl keine Informatiker sind einbezogen werden.
\lstinputlisting[language=Ruby]{chapter/test_cucumber.rb}
Nat�rlich m�ssen die Features und Szenarien auch automatisiert abgearbeitet werden. dazu findet ein Pattern Matching statt. Trifft eine Regel zu wird der definierte Kurs dieser Regel ausgef�hrt: Hier sind die Regeln und das Pattern Matching zu sehen.
\lstinputlisting[language=Ruby]{chapter/test_pattern.rb}